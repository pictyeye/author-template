%%                       Template LaTeX pour SSTIC
%%
%%
%% Utilisation
%% ===========
%%
%% Pour compiler les sources, c'est aussi simple que :
%%     $ make
%%
%% Le résultat est dans sstic-actes.pdf
%%
%% Ce modèle doit être la base commune à tous les articles LaTeX pour
%% SSTIC.
%%
%% Un exemple d'article est disponible dans le répertoire MonNom/ En
%% tant qu'auteur, vous devez copier ce répertoire et l'adapter.
%%
%% Important : Vous ne devez pas pas toucher aux fichiers préfixés par _
%%

\documentclass[b5paper,11pt]{llncs}

%% Nous avons déjà sélectionné quelques packages, si vous avez besoin
%% d'ajouter un nouveau package, veuillez contacter l'équipe (actes
%% (AT) lists.sstic.org) car nous devons vérifier la compatibilité des
%% modules entre eux.
%%

%%%%%%%%%
% Babel %
%%%%%%%%%
\newif\ifFBSmallCapsFigTabCaptions  \FBSmallCapsFigTabCaptionstrue
\usepackage[frenchb,english]{babel}
\frenchbsetup{SmallCapsFigTabCaptions=false}
% Bug:
% babel wrongly warns about Figure/Table captions being possibly wrong in French
% Since the warning does not apply here (false positive), we can remove it.
\frenchbsetup{SuppressWarning=true}


%%%%%%%%%%%%%%%%%%%%%%%%%%%
% Fonts and similar stuff %
%%%%%%%%%%%%%%%%%%%%%%%%%%%
\usepackage[utf8x]{inputenc}
\usepackage[fit]{truncate}
\usepackage{xspace}
\usepackage{ean13isbn}     % Requires the texlive-fonts-extra Debian package
\usepackage[T1]{fontenc}
\usepackage{microtype}     % Requires the cm-super Debian package
\usepackage{lmodern}
\usepackage{eurosym}
\usepackage{amsmath}
\usepackage{amssymb}
\usepackage{pifont}


%%%%%%%%%%%%%%%%%%%%%%%%
% Graphics, listing... %
%%%%%%%%%%%%%%%%%%%%%%%%
\usepackage{fancyhdr}
\usepackage[table]{xcolor}
\usepackage{graphicx}
\usepackage{pgfplots}
\pgfplotsset{compat=1.11}
\usepackage{listings}
\usepackage{tabularx}
\usepackage{array}
\usepackage{longtable}
\usepackage{multirow}
\usepackage{lscape}
\usepackage{etoolbox}
\usepackage{subfigure}
\usepackage{verbatim}
\usepackage{pmboxdraw}
\usepackage{fancyvrb}
\usepackage{algorithm}
\usepackage[noend]{algpseudocode}
\usepackage{tcolorbox}
\usepackage{booktabs}
\usepackage{rotating}

%%%%%%%%
% TikZ %
%%%%%%%%
\usepackage{tikz}
\usepackage{tikz-qtree}
\usetikzlibrary{arrows,positioning}
\usetikzlibrary{patterns}
\usetikzlibrary{automata}
\usetikzlibrary{calc}
\usepackage[fancy]{tikz-inet}
\usetikzlibrary{shapes,backgrounds}
\usetikzlibrary{trees}


%%%%%%%%%%%%%%%%%%%%%%%%%%%%%%
% Biblio, index and acronyms %
%%%%%%%%%%%%%%%%%%%%%%%%%%%%%%
\usepackage{makeidx}
\makeindex
\usepackage[nolist]{acronym}
\usepackage[sort]{cite}


%%%%%%%%
% Misc %
%%%%%%%%
\usepackage{alltt}
\usepackage{ifthen}
\usepackage[hyphens]{url}
\usepackage[super]{nth}
\usepackage[binary-units=true]{siunitx}


\renewcommand\andname{et}
\renewcommand\lastandname{ et}

\def\keywordname{{\bf Mots-cl\'es:}}
\def\ackname{Remerciements.}
\def\andname{et}
\newcommand{\keywords}[1]{\par\addvspace\baselineskip\noindent\keywordname\enspace\ignorespaces#1}


\definecolor{grigri}{RGB}{233,232,234}
\lstset{breaklines=true,
        tabsize=4,
        boxpos=c,
        showstringspaces=false,
        basicstyle=\ttfamily\scriptsize,
        aboveskip=\bigskipamount,
        belowskip=\bigskipamount,
        captionpos=b,
        language=C,
        extendedchars=false
        basicstyle=\scriptsize\sffamily\color{darkgray},
        keywordstyle=\bfseries\color{purple},
        identifierstyle=\bfseries\color{black},
        commentstyle=\ttfamily\itshape\color{purple},
        stringstyle=\ttfamily\color{brown},
        showstringspaces=false,
        frame=leftline,
        backgroundcolor=\color{grigri},
        fillcolor=\color{grigri}}

\lstdefinelanguage{Rust}%
  {
   morekeywords={abstract,alignof,as,become,box,%
                 break,const,continue,crate,do,%
                 else,enum,extern,false,final,%
                 fn,for,if,impl,in,%
                 let,loop,macro,match,mod,%
                 move,mut,offsetof,override,priv,%
                 proc,pub,pure,ref,return,%
                 Self,self,sizeof,static,struct,%
                 super,trait,true,type,typeof,%
                 unsafe,unsized,use,virtual,where,%
                 while,yield},%
   sensitive,%
   morekeywords=[1]{\$},%
   morecomment=[s]{/*}{*/},%
   morecomment=[l]//,%
   morestring=[b]",%
%   morestring=[b]', Unfortunately lifetimes also use this and it
%   breaks
   classoffset=1,
   morekeywords={u8,u16,u32},keywordstyle=\color{blue},
   classoffset=0,
}[keywords,comments,strings]

\lstloadlanguages{[Sharp]C,[x86masm]Assembler,java,PHP,C,Python}

%%%%%%%%%%%%%%%%%%%%%%%%%%%%%%%%%%%%%%%%%%%%%%%%%%%%%%%%%%%%%%%%%%%%%%%%%%%%%%%%

% example d'usage:
%     \settitle[Mon titre court]{Mon titre super mégalong de la mort}
\newcommand{\theshorttitle}{}
\newcommand\settitle[2][]{%
 \title{#2}%
 \ifthenelse{\equal{#1}{}}%
  {\renewcommand{\theshorttitle}{#2}\toctitle{#2}\titlerunning{#2}}%
  {\renewcommand{\theshorttitle}{#1}\toctitle{#1}\titlerunning{#1}}%
}

\newcommand{\theshortauthor}{}
\newcommand\setauthor[2][]{%
 \author{#2}%
 \ifthenelse{\equal{#1}{}}%
  {\renewcommand{\theshortauthor}{#2}\tocauthor{#2}\authorrunning{#2}}%
  {\renewcommand{\theshortauthor}{#1}\tocauthor{#1}\authorrunning{#1}}%
}

\pagestyle{fancy}
\fancyhf{}
\setlength{\headheight}{16pt}
\fancyhead[LE,RO]{\thepage}
\fancyhead[RE]{\truncate{.90\headwidth}{\theshorttitle}}
\fancyhead[LO]{\truncate{.90\headwidth}{\theshortauthor}}

\newcommand{\myclearpage}{\newpage{\pagestyle{empty}\cleardoublepage}}

%%%%%%%%%%%%%%%%%%%%%%%%%%%%%%%%%%%%%%%%%%%%%%%%%%%%%%%%%%%%%%%%%%%%%%%%%%%%%%%%

\bibliographystyle{plain}

\sloppy

\begin{document}
\selectlanguage{french}

\pagestyle{empty}
\frontmatter

\chapter*{Pr\'eface}


\lipsum[1-5]

\begin{flushright}
Bon symposium,

Gérard Menvussa, pour le comit{\'e} d'Organisation.
\end{flushright}

%%% Local Variables:
%%% mode: TeX-PDF
%%% TeX-master: "../_master"
%%% End:

\vfill

%%% Local Variables:
%%% mode: TeX-PDF
%%% TeX-master: "../_master"
%%% End:


\myclearpage

\section*{Comit\'e d'organisation}
\begin{tabular}{@{}p{5cm}@{}p{6.5cm}@{}}
  Nicolas  \textsc{Bareil}        & Airbus              \\
  Mathieu  \textsc{Blanc}         & CEA/DAM             \\
  Pierre   \textsc{Capillon}      & ANSSI               \\
  Olivier  \textsc{Courtay}       & DGA-MI              \\
  Olivier  \textsc{Levillain}     & ANSSI               \\
  Benjamin \textsc{Morin}         & ANSSI               \\
  Nicolas  \textsc{Prigent}       & LSTI                \\
  Raphaël  \textsc{Rigo}          & Airbus              \\
  Sarah  \textsc{Zennou}          & Airbus              \\
  Frédéric  \textsc{Tronel}   & CentraleSupélec         \\
\end{tabular}

\section*{Comité de programme}
\begin{tabular}{@{}p{5cm}@{}p{6.5cm}@{}}
  Damien \textsc{Aumaitre}        & Quarkslab           \\
  Nicolas  \textsc{Bareil}        & Airbus              \\
  Mathieu \textsc{Blanc}          & CEA/DAM             \\
  Aurélien \textsc{Bordes}        & ANSSI               \\
  Jean-Marie \textsc{Borello}     & DGA-MI              \\
  Pierre   \textsc{Capillon}      & ANSSI               \\
  Olivier \textsc{Courtay}        & DGA-MI              \\
  Marion \textsc{Daubignard}      & ANSSI               \\
  Géraud \textsc{De Drouas}       & ANSSI               \\
  Fabrice \textsc{Desclaux}       & CEA/DAM             \\
  Isabelle \textsc{Kraemer}       & Orange              \\
  Colas \textsc{Le Guernic}       & DGA-MI              \\
  Olivier \textsc{Levillain}      & ANSSI               \\
  Thierry \textsc{Marinier}       & Lexfo               \\
  Xavier \textsc{Mehrenberger}    & Airbus              \\
  Benjamin \textsc{Morin}         & ANSSI               \\
  Camille \textsc{Mougey}         & CEA/DAM             \\
  Sarah \textsc{Nataf}            & Orange              \\
  Nicolas \textsc{Prigent}        & LSTI                \\
  Raphaël \textsc{Rigo}           & Airbus              \\
  Tiphaine \textsc{Romand-Latapie} & Airbus             \\
  Philippe \textsc{Teuwen}        & Quarkslab           \\
  Frédéric \textsc{Tronel}        & CentraleSupélec     \\
  Sarah \textsc{Zennou}           & Airbus              \\
\end{tabular}

\myclearpage

\section*{Partenaires}

\begin{center}
ANSSI -  CEA - EADS - MISC/Le journal de la s{\'e}curit{\'e} informatique - OSSIR - Sup\'elec - Universit\'e de Rennes 1
\end{center}

\begin{figure}[h]
\begin{center}
\parbox{3cm}{\includegraphics[width=3cm]{_images/anssi}}
\hfill
%\hspace*{1.5cm}
\parbox{3cm}{\includegraphics[width=3cm]{_images/cea}}
\hfill
%\hspace*{1.5cm}
\parbox{3cm}{\includegraphics[width=3cm]{_images/eads}}
\end{center}
\vfill
\begin{center}
\parbox{3cm}{\includegraphics[width=3cm]{_images/technicolor}}
\hfill
\parbox{3cm}{\includegraphics[width=3cm]{_images/inria-rennes}}
\hfill
\parbox{3cm}{\includegraphics[width=3cm]{_images/misc}}
\end{center}
\vfill
\begin{center}
\parbox{3cm}{\includegraphics[width=3cm]{_images/uni-rennes1}}
\hfill
\parbox{3cm}{\includegraphics[width=3cm]{_images/supelec}}
\hfill
\parbox{3cm}{\includegraphics[width=3cm]{_images/ossir}}
\end{center}
\vfill
\begin{center}
\end{center}
\end{figure}

%%% Local Variables:
%%% mode: TeX-PDF
%%% TeX-master: "../master"
%%% End:

\myclearpage

%%% Local Variables:
%%% mode: TeX-PDF
%%% TeX-master: "../_master"
%%% End:


\tableofcontents
\mainmatter
\pagestyle{fancy}


\renewcommand{\thelstlisting}{\arabic{lstlisting}}
\part*{Conférences}
\addcontentsline{toc}{part}{Conférences}

\newcommand{\inputarticle}[1]{%
\bgroup
\part*{Conférences}
\addcontentsline{toc}{part}{Conférences}

\newcommand{\inputarticle}[1]{%
\bgroup
\part*{Conférences}
\addcontentsline{toc}{part}{Conférences}

\newcommand{\inputarticle}[1]{%
\bgroup
\input{#1/master}
\myclearpage
\egroup
\renewcommand{\thesection}{\arabic{section}}
}

%%%%%%%%%%%%%%%%%%%%%%%%%%%%%%%%%%%%%%%%%%%%%%%%%%%%%%%%%%%%%%%%%%%%%%%%%%%%%%%

\inputarticle{MonNom}

%%%%%%%%%%%%%%%%%%%%%%%%%%%%%%%%%%%%%%%%%%%%%%%%%%%%%%%%%%%%%%%%%%%%%%%%%%%%%%%

%%% Local Variables:
%%% mode: TeX-PDF
%%% TeX-master: "master"
%%% End:

\myclearpage
\egroup
\renewcommand{\thesection}{\arabic{section}}
}

%%%%%%%%%%%%%%%%%%%%%%%%%%%%%%%%%%%%%%%%%%%%%%%%%%%%%%%%%%%%%%%%%%%%%%%%%%%%%%%

\inputarticle{MonNom}

%%%%%%%%%%%%%%%%%%%%%%%%%%%%%%%%%%%%%%%%%%%%%%%%%%%%%%%%%%%%%%%%%%%%%%%%%%%%%%%

%%% Local Variables:
%%% mode: TeX-PDF
%%% TeX-master: "master"
%%% End:

\myclearpage
\egroup
\renewcommand{\thesection}{\arabic{section}}
}

%%%%%%%%%%%%%%%%%%%%%%%%%%%%%%%%%%%%%%%%%%%%%%%%%%%%%%%%%%%%%%%%%%%%%%%%%%%%%%%

\inputarticle{MonNom}

%%%%%%%%%%%%%%%%%%%%%%%%%%%%%%%%%%%%%%%%%%%%%%%%%%%%%%%%%%%%%%%%%%%%%%%%%%%%%%%

%%% Local Variables:
%%% mode: TeX-PDF
%%% TeX-master: "master"
%%% End:


\part*{Conférences}
\addcontentsline{toc}{part}{Conférences}

\newcommand{\inputarticle}[1]{%
\bgroup
\part*{Conférences}
\addcontentsline{toc}{part}{Conférences}

\newcommand{\inputarticle}[1]{%
\bgroup
\part*{Conférences}
\addcontentsline{toc}{part}{Conférences}

\newcommand{\inputarticle}[1]{%
\bgroup
\input{#1/master}
\myclearpage
\egroup
\renewcommand{\thesection}{\arabic{section}}
}

%%%%%%%%%%%%%%%%%%%%%%%%%%%%%%%%%%%%%%%%%%%%%%%%%%%%%%%%%%%%%%%%%%%%%%%%%%%%%%%

\inputarticle{MonNom}

%%%%%%%%%%%%%%%%%%%%%%%%%%%%%%%%%%%%%%%%%%%%%%%%%%%%%%%%%%%%%%%%%%%%%%%%%%%%%%%

%%% Local Variables:
%%% mode: TeX-PDF
%%% TeX-master: "master"
%%% End:

\myclearpage
\egroup
\renewcommand{\thesection}{\arabic{section}}
}

%%%%%%%%%%%%%%%%%%%%%%%%%%%%%%%%%%%%%%%%%%%%%%%%%%%%%%%%%%%%%%%%%%%%%%%%%%%%%%%

\inputarticle{MonNom}

%%%%%%%%%%%%%%%%%%%%%%%%%%%%%%%%%%%%%%%%%%%%%%%%%%%%%%%%%%%%%%%%%%%%%%%%%%%%%%%

%%% Local Variables:
%%% mode: TeX-PDF
%%% TeX-master: "master"
%%% End:

\myclearpage
\egroup
\renewcommand{\thesection}{\arabic{section}}
}

%%%%%%%%%%%%%%%%%%%%%%%%%%%%%%%%%%%%%%%%%%%%%%%%%%%%%%%%%%%%%%%%%%%%%%%%%%%%%%%

\inputarticle{MonNom}

%%%%%%%%%%%%%%%%%%%%%%%%%%%%%%%%%%%%%%%%%%%%%%%%%%%%%%%%%%%%%%%%%%%%%%%%%%%%%%%

%%% Local Variables:
%%% mode: TeX-PDF
%%% TeX-master: "master"
%%% End:

\end{document}

%% Publication
%% ===========
%%
%% Vous devez envoyer les *sources* à actes (AT) lists.sstic.org avant
%% la date butoir.
%%
%% L'auteur consciencieux prendre soin de corriger tous les warnings
%% avant de nous envoyer ses sources :)
%%
%% Vous pouvez envoyer une archive complète (.tar.gz), ou un patch, ou
%% un lien vers votre repository, etc.
