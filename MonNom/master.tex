\settitle[Sécurité des trotinettes]{Sécurité des trotinettes : prise de contrôle des roues à distance}


% Pas besoin d'obfusquer les adresses mails, \email s'en chargera (XXX)
\setauthor[M.~MonNom, A.~MonAutreNom]{MonPrenom MonNom \and AutrePrenom AutreNom\\
  \email{monprenom.monnom@mycorp.com\\autreprenom.autrenom@mycorp.com}}

% XXX: Gérer le cas des compagnies multiples
\institute{MyCorp}

\maketitle
\index{MonNom, M.}
\index{MonAutreNom, A.}

\begin{abstract}
  Lorem ipsum dolor sit amet, consectetur adipisicing elit, sed do
  eiusmod tempor incididunt ut labore et dolore magna aliqua. Ut enim ad
  minim veniam, quis nostrud exercitation ullamco laboris nisi ut
  aliquip ex ea commodo consequat. Duis aute irure dolor in
  reprehenderit in voluptate velit esse cillum dolore eu fugiat nulla
  pariatur. Excepteur sint occaecat cupidatat non proident, sunt in
  culpa qui officia deserunt mollit anim id est laborum.
\end{abstract}

%% Découpage en sections
%% =====================
%% 
%% La classe llcns ne définit que 4 niveaux de sections :
%% 
%% * section
%% * subsection
%% * subsubsection
%% * paragraph
%% 

\section{Introduction}

\subsection{État de l'art}

Lorem ipsum dolor sit amet, consectetur adipisicing elit, sed do
eiusmod tempor incididunt ut labore et dolore magna aliqua. Ut enim ad
minim veniam, quis nostrud exercitation ullamco laboris nisi ut
aliquip ex ea commodo consequat. Duis aute irure dolor in
reprehenderit in voluptate velit esse cillum dolore eu fugiat nulla
pariatur. Excepteur sint occaecat cupidatat non proident, sunt in
culpa qui officia deserunt mollit anim id est laborum. Lorem ipsum
dolor sit amet, consectetur adipisicing elit, sed do eiusmod tempor
incididunt ut labore et \emph{dolore} magna aliqua. Ut enim ad minim
veniam, quis nostrud exercitation ullamco laboris nisi ut aliquip ex
ea commodo consequat. Duis aute irure dolor in reprehenderit in
voluptate velit esse cillum dolore eu fugiat nulla pariatur. Excepteur
sint occaecat cupidatat non proident, sunt in culpa qui officia
deserunt mollit anim id est laborum.

Je sais aussi faire des citations, dans l'article de Charlie
Lembrouille, \cite{charlielembrouille}, il est démontré que, je cite,
\og{}les trotinettes sont vulnérables aux attaques par XSS\fg{}. C'est
moche, l'image~\ref{fig:monnom:archi} en étant la preuve !

%% Images, figures
%% ===============
%% 
%% Les images doivent être placées dans un environnement figure de
%% façon à pouvoir y associer facilement une légende et un label.
%% 
%% Quelques notes:
%% 
%% * La légende doit être en dessous de l'image
%% 
%% * Le placement de l'image par LaTeX ne garantit pas que votre
%%   image sera exactement où vous souhaitiez. Remplacez donc les
%%   formules "l'image ci-dessous" mais plutôt par sa référence
%%   ("l'image~\ref{fig:monnom:archi}")
%% 
%% * Format de l'image
%%   - Compatible pdflatex : PNG, PDF
%%   - Les formats vectoriels (comme PDF) sont recommandés car il
%%     arrive de devoir redimensionner les images à l'édition
%%     finale
%% 
%% * Il est demandé de créér un répertoire img/ dédié à toutes les
%%   images 
%% 
%% * Prenez soin de vos couleurs, SSTIC n'imprime qu'en dégradé de
%%   gris, votre choix de couleurs doit donc respecter cette
%%   contrainte.
%% 
%%   Votre article sera publié sur papier et sur notre site Web. Vous
%%   pouvez donc fournir les images dans les deux "formats". Nommez
%%   explicitement vos images avec le préfixe bw- (exemple: bw-archi.pdf)
%% 
%%   À défaut d'avoir une version papier, nous la convertirons à
%%   l'aide la commande suivante:
%%       $ convert archi.pdf -colorspace Gray bw-archi.pdf
%% 

\begin{figure}[ht]
  \centering
  \includegraphics[width=0.4\textwidth]{MonNom/img/archi}
  \caption{Légende de l'image}
  \label{fig:monnom:archi}
\end{figure}

Lorem ipsum dolor sit amet, consectetur adipisicing elit, sed do
eiusmod tempor incididunt ut labore et dolore magna aliqua. Ut enim ad
minim veniam, quis nostrud exercitation ullamco laboris nisi ut
aliquip ex ea commodo consequat. Duis aute irure dolor in
reprehenderit in voluptate velit esse cillum dolore eu fugiat nulla
pariatur. Excepteur sint occaecat cupidatat non proident, sunt in
culpa qui officia deserunt mollit anim id est laborum. Lorem ipsum
dolor sit amet, consectetur adipisicing elit, sed do eiusmod tempor
incididunt ut labore et dolore magna aliqua. Ut enim ad minim veniam,
quis nostrud exercitation ullamco laboris nisi ut aliquip ex ea
commodo consequat. Duis aute irure dolor in reprehenderit in voluptate
velit esse cillum dolore eu fugiat nulla pariatur. Excepteur sint
occaecat cupidatat non proident, sunt in culpa qui officia deserunt
mollit anim id est laborum.

Avec des listes partout :

\begin{itemize}
\item Lorem ipsum dolor sit amet
  \begin{itemize}
  \item Parce que je le vaux bien
  \item N'est-ce pas ?
  \end{itemize}
\item Consectetur adipisicing elit
\item Sed do eiusmod tempor
\end{itemize}

Lorem ipsum dolor sit amet, consectetur adipisicing elit, sed do
eiusmod tempor incididunt ut labore et dolore magna aliqua. Ut enim ad
minim veniam, quis nostrud exercitation ullamco laboris nisi ut
aliquip ex ea commodo consequat. Duis aute irure dolor in
reprehenderit in voluptate velit esse cillum dolore eu fugiat nulla
pariatur. Excepteur sint occaecat cupidatat non proident, sunt in
culpa qui officia deserunt mollit anim id est laborum. Lorem ipsum
dolor sit amet, consectetur adipisicing elit, sed do eiusmod tempor
incididunt ut labore et dolore magna aliqua. Ut enim ad minim veniam,
quis nostrud exercitation ullamco laboris nisi ut aliquip ex ea
commodo consequat. Duis aute irure dolor in reprehenderit in voluptate
velit esse cillum dolore eu fugiat nulla pariatur. Excepteur sint
occaecat cupidatat non proident, sunt in culpa qui officia deserunt
mollit anim id est laborum.

\section{Use the source, Luke}

%% Texte verbatim
%% ==============
%% 
%% lstlisting
%% ----------
%% 
%% Documentation complète :  http://mirrors.ctan.org/macros/latex/contrib/listings/listings.pdf
%% 
%% Options utiles :
%% 
%% * ``language'' peut prendre les valeurs suivantes:
%%    Python	Java		PHP		[x86masm]Assembler
%%    C		Perl		HTML		[Sharp]C
%% 
%% * ``numbers'' pour numéroter les ligne (``stepnumbers'' pour  contrôler l'incrément)
%% 
%% * ``basicstyle'' pour modifier la police utilisée (exemple: basicstyle=\tiny)
%% 

Des fois, je peux aussi inclure du texte brut, comme la sortie
standard en figure~\ref{lst:monnom:outhello} qui a été généré par le
code~\ref{lst:monnom:helloword}.

\begin{lstlisting}[language={},caption={Sortie standard},label={lst:monnom:outhello}]
Hello world everybody
\end{lstlisting}

\begin{lstlisting}[language={Python},caption={Mon premier code},label={lst:monnom:helloword}]
#! /usr/bin/env python

import sys

def hello(name):
    print 'Hello world %s' % (name)

if __name__ == '__main__':
    hello(sys.argv[1] if len(sys.argv) > 1 else 'everybody')
\end{lstlisting}


%%% Local Variables:
%%% mode: TeX-PDF
%%% TeX-master: "master"
%%% End:
