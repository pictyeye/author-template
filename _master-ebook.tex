%%                       Template LaTeX pour SSTIC
%% 
%% 
%% Utilisation
%% ===========
%% 
%% Pour compiler les sources, c'est aussi simple que :
%%     $ make sstic-actes.html
%% 
%% Ce modèle doit être la base commune à tous les articles LaTeX pour
%% SSTIC.
%% 
%% Un exemple d'article est disponible dans le répertoire MonNom/ En
%% tant qu'auteur, vous devez copier ce répertoire et l'adapter.
%% 
%% Important : Vous ne devez pas pas toucher aux fichiers préfixés par _
%% 

\documentclass[b5paper,11pt]{llncs}

%% Nous avons déjà sélectionné quelques packages, si vous avez besoin
%% d'ajouter un nouveau package, veuillez contacter l'équipe (actes
%% (AT) lists.sstic.org) car nous devons vérifier la compatibilité des
%% modules entre eux.
%% 

\usepackage[utf8]{inputenc}
\usepackage{textcomp}
\usepackage[T1]{fontenc}
\usepackage[french]{babel}
\usepackage[fit]{truncate}
\usepackage{xspace}
\usepackage[table]{xcolor}
\usepackage{graphicx}
\usepackage{fancyhdr}
\usepackage{listings}
\usepackage{alltt}
\usepackage{ifthen}
\usepackage{url}

\DeclareUnicodeCharacter{00A0}{~}

% Hack to have agenda working
\newcommand{\pause}[1]{\emph{#1}}

% Needed by a lot of people
\usepackage{tikz}
\usepackage{tikz-qtree}
\usetikzlibrary{arrows,positioning}
\usetikzlibrary{patterns}
\usetikzlibrary{shapes}

% htlatex uses latex in dvi mode so force the extension to something (either png or eps)
\DeclareGraphicsExtensions{.eps, .png}

\usepackage{makeidx}
\makeindex

\usepackage{tabularx}
\usepackage{array}
\usepackage{multirow}
\usepackage{lscape}

\usepackage{subfigure}

\usepackage{etoolbox}
\usetikzlibrary{shapes,backgrounds}

\usetikzlibrary{trees}

\usepackage{microtype}     % Needs the cm-super Debian package

\usepackage{verbatim}
\usepackage{amsmath}
\usepackage{pifont}
\usepackage{pmboxdraw}
\usepackage{fancyvrb}

\renewcommand\andname{et}
\renewcommand\lastandname{ et}

\newcommand{\xxx}{{\bf \LARGE \textcolor{red}{XXX}}\xspace}
\newcommand{\XXX}{\xxx}
\newtheorem{exx}{Exemple}

\def\keywordname{{\bf Mots-cl\'es:}}
\def\ackname{Remerciements.}
\def\andname{et}
\newcommand{\keywords}[1]{\par\addvspace\baselineskip\noindent\keywordname\enspace\ignorespaces#1}


\definecolor{grigri}{RGB}{233,232,234}
\lstset{breaklines=true,
        tabsize=4,
        boxpos=c,
        showstringspaces=false,
        basicstyle=\ttfamily\scriptsize,
        aboveskip=\bigskipamount,
        belowskip=\bigskipamount,
        captionpos=b,
        language=C,
        extendedchars=false
        basicstyle=\scriptsize\sffamily\color{darkgray},
        keywordstyle=\bfseries\color{purple},
        identifierstyle=\bfseries\color{black},
        commentstyle=\ttfamily\itshape\color{purple},
        stringstyle=\ttfamily\color{brown},
        showstringspaces=false,
        frame=leftline,
        backgroundcolor=\color{grigri},
        fillcolor=\color{grigri}}

\lstloadlanguages{[Sharp]C,[x86masm]Assembler,java,PHP,C,Python}

%%%%%%%%%%%%%%%%%%%%%%%%%%%%%%%%%%%%%%%%%%%%%%%%%%%%%%%%%%%%%%%%%%%%%%%%%%%%%%%%

% example d'usage:
%     \settitle[Mon titre court]{Mon titre super mégalong de la mort}
\newcommand{\theshorttitle}{}
\newcommand\settitle[2][]{%
 \title{#2}%
 \ifthenelse{\equal{#1}{}}%
  {\renewcommand{\theshorttitle}{#2}\toctitle{#2}\titlerunning{#2}}%
  {\renewcommand{\theshorttitle}{#1}\toctitle{#1}\titlerunning{#1}}%
}

\newcommand{\theshortauthor}{}
\newcommand\setauthor[2][]{%
 \author{#2}%
 \ifthenelse{\equal{#1}{}}%
  {\renewcommand{\theshortauthor}{#2}\tocauthor{#2}\authorrunning{#2}}%
  {\renewcommand{\theshortauthor}{#1}\tocauthor{#1}\authorrunning{#1}}%
}

\newcommand{\myclearpage}{\newpage{\pagestyle{empty}\cleardoublepage}}

%%%%%%%%%%%%%%%%%%%%%%%%%%%%%%%%%%%%%%%%%%%%%%%%%%%%%%%%%%%%%%%%%%%%%%%%%%%%%%%%

\bibliographystyle{plain}


\begin{document}

\tableofcontents

\part*{Conférences}
\addcontentsline{toc}{part}{Conférences}

\newcommand{\inputarticle}[1]{%
\bgroup
\part*{Conférences}
\addcontentsline{toc}{part}{Conférences}

\newcommand{\inputarticle}[1]{%
\bgroup
\part*{Conférences}
\addcontentsline{toc}{part}{Conférences}

\newcommand{\inputarticle}[1]{%
\bgroup
\input{#1/master}
\myclearpage
\egroup
\renewcommand{\thesection}{\arabic{section}}
}

%%%%%%%%%%%%%%%%%%%%%%%%%%%%%%%%%%%%%%%%%%%%%%%%%%%%%%%%%%%%%%%%%%%%%%%%%%%%%%%

\inputarticle{MonNom}

%%%%%%%%%%%%%%%%%%%%%%%%%%%%%%%%%%%%%%%%%%%%%%%%%%%%%%%%%%%%%%%%%%%%%%%%%%%%%%%

%%% Local Variables:
%%% mode: TeX-PDF
%%% TeX-master: "master"
%%% End:

\myclearpage
\egroup
\renewcommand{\thesection}{\arabic{section}}
}

%%%%%%%%%%%%%%%%%%%%%%%%%%%%%%%%%%%%%%%%%%%%%%%%%%%%%%%%%%%%%%%%%%%%%%%%%%%%%%%

\inputarticle{MonNom}

%%%%%%%%%%%%%%%%%%%%%%%%%%%%%%%%%%%%%%%%%%%%%%%%%%%%%%%%%%%%%%%%%%%%%%%%%%%%%%%

%%% Local Variables:
%%% mode: TeX-PDF
%%% TeX-master: "master"
%%% End:

\myclearpage
\egroup
\renewcommand{\thesection}{\arabic{section}}
}

%%%%%%%%%%%%%%%%%%%%%%%%%%%%%%%%%%%%%%%%%%%%%%%%%%%%%%%%%%%%%%%%%%%%%%%%%%%%%%%

\inputarticle{MonNom}

%%%%%%%%%%%%%%%%%%%%%%%%%%%%%%%%%%%%%%%%%%%%%%%%%%%%%%%%%%%%%%%%%%%%%%%%%%%%%%%

%%% Local Variables:
%%% mode: TeX-PDF
%%% TeX-master: "master"
%%% End:


\end{document}

%% Publication
%% ===========
%% 
%% Vous devez envoyer les *sources* à actes (AT) lists.sstic.org avant
%% la date butoir.
%% 
%% L'auteur consciencieux prendre soin de corriger tous les warnings
%% avant de nous envoyer ses sources :)
%%
%% Vous pouvez envoyer une archive complète (.tar.gz), ou un patch, ou
%% un lien vers votre repository, etc.
%% 
%%% Local Variables:
%%% mode: TeX-PDF
%%% TeX-master: "master"
%%% End:
